\chapter{Gerenciamento do projeto}

O objetivo deste capítulo é descrever com detalhes a parte de gerenciamento do projeto utilizando como base o PMBOK \cite{pmbok} para a criação dos planos de gerenciamento.

\section{Recursos Humanos}

O plano de gerenciamento de recursos humanos visa estabelecer quais os recursos humanos serão utilizados durante o projeto.

\subsection{Origem dos recursos}

O projeto é formado por um grupo de 14 pessoas, no qual temos estudantes das 5 engenharias: Engenharia de Software, Engenharia Automotiva, Engenharia Aeroespacial, Engenharia Eletronica e Engenharia de Energia. O grupo é dividido de forma técnica em grupo de Estruturas, grupo de Automação e grupo de Software. No total, são três grupos técnicos.

\subsection{Equipe e Responsabilidades}

\begin{table}[ht]
    \centering
	\label{equipe}
    \caption{Equipe e Responsabilidades}
    \begin{tabular}{ |c|c|c|c|c } 
        \hline
        Matrícula & Nome & Engenharia & Papel \\
        \hline
        13/0039217 & Wesley Pereira Araujo & Software & Diretor Técnico \\
        13/0136484 & Victor Hugo Arnaud Deon & Software & Desenvolvedor \\
        15/0045603 & Renata Francelino de Souza & Software & Desenvolvedora \\
        14/0031634 & Stefânia Bezerra da Silva & Software & Desenvolvedora \\
        13/0113107 & Guilherme Henrique Rodrigues Vaz & Aeroespacial & Diretor de Qualidade\\
        13/0119237 & Laís Rocha Carvalho & Aeroespacial & Diretora Técnica \\
        13/0008206 & Diego Ribeiro Borges Rasmussen & Aeroespacial & Desenvolvedor \\
        15/0039611 & Laís Almeida Nunces & Aeroespacial & Desenvolvedora \\
        13/0013226 & Lucas Ferreira Rossi & Aeroespacial & Desenvolvedor \\
        15/0004206 & Abia Matos de Oliveira & Energia & Desenvolvedora \\
        15/0020341 & Priscilla Costa de Souza & Eletrônica & Diretora Geral\\
        13/0018155 & Victor Barreto Batalha & Eletrônica & Desenvolvedor\\
        15/0041560 & Luiza Carneiro Cezário & Eletrônica & Diretora Técnica \\
        15/0041527 & Luiz Henrique Rocha Marinho & Eletrônica & Desenvolvedor \\
        \hline
    \end{tabular}
\end{table}

\subsection{Professores}

\begin{itemize}
    \item Alex Reis (Engenharia de Energia)
    \item Rhander Viana (Engenharia Automotiva)
    \item Sebastien R. M. J. Rondineau (Engenharia Aeroespacial)
    \item Ricardo M. Chaim (Engenharia de Software)
    \item Guillermo A. Bestard (Engenharia Eletrônica)
\end{itemize}

\section{Tempo}

O plano de gerenciamento de tempo tem como finalidade estabelecer como será o gerenciamento do tempo do projeto, mostrando o cronograma e suas respectivas atividades.

\subsection{Definição das atividades - EAP}

Esse processo busca identificar e documentar as ações específicas a serem realizadas para produzir as entregas do projeto. O foco é exatamente dividir o trabalho em pacotes contendo um conjunto de atividades de alguma forma relacionadas ou interdependentes.

A Estrutura Analítica do Projeto ou EAP, foi utilizada como entrada do processo fornecendo os pacotes de trabalho previamente discutidos. Esses pacotes foram então decompostos de maneira a gerar suas respectivas atividades.

A imagem da EAP se encontra no apêndice \ref{apendice:eap}

\subsection{Cronograma}

Após adquirir conhecimento sobre as atividades que devem ser feitas e suas relações de dependência e estimar as durações das mesmas, deve-se sequenciá-las em um documento que consiste no cronograma. O mesmo é extremamente importante para dar uma visibilidade geral sobre o projeto. Além disso, o cronograma consiste na principal forma de monitoramento e controle referente ao escopo de tempo do projeto.

A imagem do cronograma se encontra no apêndice \ref{apendice:cronograma}

\section{Aquisição}

O objetivo deste plano é descrever formalmente o processo de aquisição de recursos, tanto de bens quanto serviços, para o projeto. Contendo planejamento, descrição, evolução e custo, de cada uma das propostas de compra, que se fizerem necessárias, até a sua real aquisição.

\subsection{Processo}

Inicialmente vamos identificar as necessidades de aquisição, uma vez identificada a necessidade de aquisição de um determinado bem ou serviço para o projeto, o membro que fez esta identificação deve relatar aos demais integrantes a necessidade desta aquisição e justificar a escolha da mesma. Com isso há um levantamento de contra proposta, ou seja, após receber a solicitação da aquisição necessária para o projeto, os demais membros deverão procurar por recursos concorrentes que atendam à necessidade levantada na etapa anterior além de fornecedores alternativos.


\subsection{Recursos}

Os custos totais do projeto serão fornecidos em função do planejamento dos materiais utilizados por suas respectivas áreas para que, assim, seja realizada uma melhor análise. Foi acordado entre os membros que, inicialmente, cada um deveria realizar uma contribuição de R\$ 50,00 juntando, assim, um valor inicial de R\$ 700,00.

\subsection{Registro}
\label{subsec:Registro}

Com a a aquisição já definida, será realizado o acompanhamento da aquisição e registro desta.

\begin{table}[ht]
    \centering
	\label{aqeletronica}
    \caption{Registro de Aquisição}
    \begin{tabular}{|p{2.0in}|p{0.8in}|p{0.7in}|p{1.0in}|}
        \hline
        Item & Quantidade & Valor & Total \\
        \hline
        Sensor ultrasônico HC-SR04 &  3 & R\$ 15,00 & R\$ 75,00 \\
        Motor Vibracall & 4 & R\$ 13,00 & R\$ 52,00 \\
        Raspberry Pi 3 Model B & 1 & R\$ 210,00 & R\$ 210,00 \\
        MSP430 & 1 & R\$ 45,00 & R\$ 45,00 \\
        Camera Raspberry Pi 5MP & 1 & R\$ 50,00 & R\$ 50,00 \\
        Conta Google Play & 1 & R\$ 98,00 & R\$ 98,00 \\
        Metro Quadrado de Manta de fibra de vidro com resina & 1 & R\$ 100,00 & R\$ 100,00 \\
        Vara de Pescar & 1 & R\$ 50,00 & R\$ 50,00 \\
        Chapa de alumínio & 1 & R\$ 20,00 & R\$ 20,00 \\
        Imãs de Neodímio & 8 & R\$ 12,50 & R\$ 100,00 \\
        Metro de fiação & 3 & R\$ 10,00  & R\$ 30,00 \\
        \hline
        Total & & & R\$ 830,00 \\
        \hline
    \end{tabular}
\end{table}

\section{Riscos}

O plano de gerenciamento de riscos como objetivo a identificação dos possíveis riscos envolvidos no projeto, tal como os planos para tratamento.


\subsection{Riscos do projeto}

Os riscos do projeto se encontram na tabela \ref{riscos}

\begin{table}[h]
    \centering
    \caption{Riscos do projeto} \label{riscos}
    \begin{tabular}{p{0.5in}p{1.5in}p{1.5in}p{1.5in}}
    
    ID & Risco & Causa & Efeito\\ \hline
    RP1 & Falta de componentes de outros subsistemas & Atraso na execução do projeto de outros subsistemas & Atraso na entrega de parte do projeto \\ \hline
    RP2 & Material necessário indisponível & Falta do material em lojas próximas & Atraso na entrega de parte do projeto \\ \hline
    RP3 & Ausência de técnico no Galpão & - & Atraso na entrega de parte do projeto \\ \hline
    RP4 & Dificuldade de Integração dos sistemas & Falha na comunicação da equipe & Erro de projeto \\ \hline
    RP5 & Reparo de grande proporção & Falha inesperada na estrutura & Não entrega do projeto \\ \hline
    RP6 & Câmera não capta faixa de pedestre & Erro de processamento de imagem no algoritmo ou câmera de baixa qualidade & Não entrega da funcionalidade \\ \hline
    RP7 & Dificuldade em comunicação WiFi entre Raspberry e API & Desconhecimento da tecnologia & Não entrega de parte do projeto \\ \hline
    RP8 & Dificuldade no desenvolvimento da API pública & Inexperiência do time com a tecnologia & Não entrega da funcionalidade \\ \hline
    RP9& Alimentação do sistema mal dimensionada & Erro nos cálculos/ falta de testes & Atraso na entrega \\ \hline
    RP10 & Hardware não se adapta a estrutura estabelecida & Falha de comunicação entre os subsistemas & Atraso na entrega/ necessidade de alteração \\ \hline
    RP11 & Dificuldade de implementar um algoritmo para os sensores & Dificuldade em entender seu funcionamento & Atraso/ Não entrega do subsistema \\ \hline
    RP12 & Hardware insuficiente ou incapaz de realizar o processamento de imagem em tempo razoável & Erro ao dimensionar o microprocessador & Atraso/ Não entrega do subsistema \\ \hline 
    RP13 & Queima ou mal funcionamento de componentes & Falha inesperada/ mal uso & Atraso na entrega do subsistema \\ \hline
    \end{tabular}
\end{table}

\subsection{Medidas corretivas e preventivas}

As medidas corretivas e preventivas se encontram na tabela \ref{medidas}

\begin{table}[ht]
    \centering
    \caption{Medidas corretivas e preventivas} \label{medidas}
    \begin{tabular}{p{0.5in}p{2.3in}p{2.3in}}
    Número & Medidas Preventivas & Medidas Corretivas \\ \hline
    RP1 & Manter a comunicação com os outros subsistemas para se preparar para possíveis atrasos & Redimensionar cronograma \\ \hline
    RP2 & Procurar antecipadamente locais alternativos de distribuição do material & Procurar material alternativo que atenda aos requisitos\\ \hline
    RP3 & Agendamento prévio & Utilizar local alternativo\\ \hline
    RP4 & Testes de integração e validação semanais & Consultar especialistas (professores) \\ \hline
    RP5 & Trabalhar com prazos que permita reparo & Busca de alternativa mais simplesVOu \\ \hline
    RP6 & Pesquisa de estudos relacionados ao reconhecimento de faixa de pedestre & Desenvolvimento por reconhecimento de QR Code com informações da localização da faixa de pedestre. \\ \hline
    RP7 & Testes de integração e validação & Alterar para tecnologia bluetooth \\ \hline
    RP8 & Integrar funcionalidade na api imediatamente ao desenvolvimento & Time se reunir para a construção da api \\ \hline
    RP9 & Realizar simulações e testes antes de implementar & Substituir componentes danificados \\ \hline
    RP10 & Manter comunicação com outros subsistemas  & Adaptar a estrutura ao hardware \\ \hline
    RP11 & Estudar detalhadamente o datasheet do sensor e procurar ajuda se necessário  & Fazer testes e calibração constantes dos sensores  \\ \hline
    RP12 & Analisar detalhadamente todos os requisitos de software e processamento de imagem & Otimizar o código de processamento de imagem para reduzir o tempo de processamento.  \\ \hline
    RP13 & Adquirir componentes reserva & Conferir com frequência o funcionamento dos componentes  \\ \hline
    \end{tabular}
\end{table}