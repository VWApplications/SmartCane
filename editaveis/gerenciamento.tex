\chapter{Gerenciamento do projeto}

O objetivo deste capítulo é descrever com detalhes a parte de gerenciamento do projeto utilizando como base o \cite{pmbok} para a criação dos planos de gerenciamento.

\section{Recursos Humanos}

O plano de gerenciamento de recursos humanos visa estabelecer quais os recursos humanos serão utilizados durante o projeto.

\subsection{Origem dos recursos}

O projeto é formado por um grupo de 14 pessoas, na qual estão dividios em 5 engenharias: Engenharia de Software, Engenharia Automotiva, Engenharia Aeroespacial, Engenharia Eletronica e Engenharia de Energia.

\subsection{Equipe e Responsabilidades}

\begin{table}[ht]
    \centering
	\label{equipe}
    \caption{Equipe e Responsabilidades}
    \begin{tabular}{ |c|c|c|c|c } 
        \hline
        Matrícula & Nome & Engenharia & Papel \\
        \hline
        13/0039217 & Wesley Pereira Araujo & Software & Diretor Técnico \\
        13/0136484 & Victor Hugo Arnaud Deon & Software & Desenvolvedor \\
        15/0045603 & Renata Francelino de Souza & Software & Desenvolvedor \\
        14/0031634 & Stefânia Bezerra da Silva & Software & Desenvolvedor \\
        - & Guilherme Henrique Rodrigues Vaz & Aeroespacial & - \\
        - & Laís Rocha Carvalho & Aeroespacial & - \\
        - & Diego Ribeiro Borges Rasmussen & Aeroespacial & - \\
        - & Laís Almeida Nunces & Aeroespacial & - \\
        - & Lucas Ferreira Rossi & Aeroespacial & - \\
        - & Abia Matos de Oliveira & Energia & - \\
        - & Priscilla Costa de Souza & Eletrônica & - \\
        - & Victor Barreto Batalha & Eletrônica & - \\
        - & Luiza Carneiro Cezário & Eletrônica & - \\
        - & Luiz Henrique Rocha Marinho & Eletrônica & - \\
        \hline
    \end{tabular}
\end{table}

\section{Tempo}

O plano de gerenciamento de tempo tem como finalidade estabelecer como será o gerenciamento do tempo do projeto, mostrando o cronograma e suas respectivas atividades.

\subsection{Definição das atividades - EAP}

Esse processo busca identificar e documentar as ações específicas a serem realizadas para produzir as entregas do projeto. O foco é exatamente dividir o trabalho em pacotes contendo um conjunto de atividades de alguma forma relacionadas ou interdependentes.

A Estrutura Analítica do Projeto ou EAP, foi utilizada como entrada do processo fornecendo os pacotes de trabalho previamente discutidos. Esses pacotes foram então decompostos de maneira a gerar suas respectivas atividades.

IMAGEM DA EAP

\subsection{Cronograma}

Após adquirir conhecimento sobre as atividades que devem ser feitas e suas relações de dependência e estimar as durações das mesmas, deve-se sequenciá-las em um documento que consiste no cronograma. O mesmo é extremamente importante para dar uma visibilidade geral sobre o projeto. Além disso, o cronograma consiste na principal forma de monitoramento e controle referente ao escopo de tempo do projeto.

IMAGEM DO CRONOGRAMA OU ROADMAP

\section{Aquisição}

O objetivo deste plano é descrever formalmente o processo de aquisição de recursos, tanto de bens quanto serviços, para o projeto. Contendo planejamento, descrição, evolução e custo, de cada uma das propostas de compra, que se fizerem necessárias, até a sua real aquisição.

\subsection{Processo}

Inicialmente vamos identificar as necessidades de aquisição, uma vez identificada a necessidade de aquisição de um determinado bem ou serviço para o projeto, o membro que fez esta identificação deve relatar aos demais integrantes a necessidade desta aquisição e justificar a escolha da mesma. Com isso há um levantamento de contra proposta, ou seja, após receber a solicitação da aquisição necessária para o projeto, os demais membros deverão procurar por recursos concorrentes que atendam à necessidade levantada na etapa anterior além de fornecedores alternativos.


\subsection{Recursos}

Os custos totais do projeto serão fornecidos em função do planejamento dos materiais utilizados por suas respectivas áreas para que, assim, seja realizada uma melhor análise. Foi acordado entre os membros que, inicialmente, cada um deveria realizar uma contribuição de R\$ 50,00 juntando, assim, um valor inicial de R\$ 700,00.

\subsection{Registro}

Com a a aquisição já definida, vamos fazer o acompanhamento da aquisição e registro da mesma.

TODOS OS CUSTOS VAI FICAR EM UMA TABELA SÓ

\begin{table}[ht]
    \centering
	\label{aqeletronica}
    \caption{Registro de Aquisição}
    \begin{tabular}{|p{2.0in}|p{0.8in}|p{0.7in}|p{1.0in}|}
        \hline
        Item & Quantidade & Valor & Total \\
        \hline
        Sensor ultrasônico HC-SR04 &  3 & R\$ 15,00 & R\$ 75,00 \\
        Motor Vibracall & 4 & R\$ 13,00 & R\$ 52,00 \\
        Raspberry Pi 3 Model B & 1 & R\$ 210,00 & R\$ 210,00 \\
        MSP430 & 1 & R\$ 45,00 & R\$ 45,00 \\
        Camera para Raspberry Pi 5mp & 1 & R\$ 50,00 & R\$ 50,00 \\
        Conta Google Play & 1 & R\$ 98,00 & R\$ 98,00 \\
        \hline
        Total & & & R\$ 530,00 \\
        \hline
    \end{tabular}
\end{table}

\section{Riscos}

O plano de gerenciamento de riscos como objetivo a identificação dos possíveis riscos envolvidos no projeto, tal como os planos para tratamento.


\subsection{Riscos do projeto }

TODOS OS RISCOS TAMBÉM FICARÀ EM UMA TABELA SÓ

\begin{table}[ht]
    \centering
    \caption{Riscos do projeto}
    \begin{tabular}{p{0.5in}p{1.5in}p{1.5in}p{1.5in}}
    ID & Risco & Causa & Efeito\\ \hline
    RP1 & Falta de componentes de outros subsistemas & Atraso na execução do projeto de outros subsistemas & Atraso na entrega de parte do projeto \\ \hline
    RP2 & Material necessário indisponível & Falta do material em lojas próximas & Atraso na entrega de parte do projeto \\ \hline
    RP3 & Usinagem de alta complexidade & Alta complexidade das peças e inexperiência dos membros com o processo & Não obtenção das peças esperadas e necessidade de alteração no projeto \\ \hline
    RP4 & Ausência de técnico no Galpão & - & Atraso na entrega de parte do projeto \\ \hline
    RP5 & Dificuldade de Integração dos sistemas & Falha na comunicação da equipe & Erro de projeto \\ \hline
    RP6 & Reparo de grande proporção & Falha inesperada na estrutura & Não entrega do projeto \\ \hline
    RP7 & Câmera não capta faixa de pedestre & Erro de processamento de imagem no algoritmo ou câmera de baixa qualidade & Não entrega da funcionalidade \\ \hline
    RP8 & Dificuldade em comunicação WiFi entre Raspberry e API & Desconhecimento da tecnologia & Não entrega de parte do projeto \\ \hline
    RP9 & Dificuldade no desenvolvimento da API pública & Inexperiência do time com a tecnologia & Não entrega da funcionalidade \\ \hline
    \end{tabular}
\end{table}

\subsection{Medidas corretivas e preventivas}

\begin{table}[ht]
    \centering
    \caption{Medidas corretivas e preventivas}
    \begin{tabular}{p{0.5in}p{2.3in}p{2.3in}}
    Número & Medidas Preventivas & Medidas Corretivas \\ \hline
    RP1 & - & - \\ \hline
    RP2 & - & - \\ \hline
    RP3 & - & - \\ \hline
    RP4 & - & - \\ \hline
    RP5 & Testes de integração e validação semanais & Consultar especialistas (professores) \\ \hline
    RP6 & - & - \\ \hline
    RP7 & Pesquisa de estudos relacionados ao reconhecimento de faixa de pedestre & Desenvolvimento por reconhecimento de QR Code com informações da localização da faixa de pedestre. \\ \hline
    RP8 & Testes de integração e validação & Alterar para tecnologia bluetooth \\ \hline
    RP9 & Integrar funcionalidade na api imediatamente ao desenvolvimento & Time se reunir para a construção da api \\ \hline
    \end{tabular}
\end{table}