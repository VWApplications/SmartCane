\chapter[Requisitos do projeto]{REQUISITOS DO PROJETO}

\section{Requisitos do Sistema}

O \textit{Smart Way} será um dispositivo criado com a finalidade de auxiliar deficientes visuais a se guiarem por um caminho escolhido. Abaixo se encontram os requisitos gerais do sistema:

\begin{itemize}
\item O dispositivo irá indicar os objetos a frente por meio de vibrações;
\item O dispositivo deverá orientar o usuário com segurança até o destino final;
\end{itemize}

\section{Requisitos dos Sub Sistemas}

\subsection{Requisitos estruturais}

\begin{itemize}
\item O dispositivo será dobrável; 
\item A estrutura deverá ser rígida e resistente;
\item O dispositivo deverá ser leve;
\item O dispositivo deverá ser impermeável;
\item A estutura deverá conter um sistema de refrigeração; 
\item A pulseira deverá ser ajustável ao pulso;
\item O dispositivo irá se recolher manualmente;
\item A estutura deverá abrigar com segurança todos os componentes eletrônicos e de energia.
\end{itemize}

\subsection{Requisitos eletrônicos}

\begin{itemize}
\item O sistema embarcado deve fazer o processamento de imagem para analisar a existência de uma faixa de pedestre.
\item O sistema deve controlar toda a parte sensorial acoplada ao microprocessador.
\item O sistema comunica-se via servidor do aplicativo através de uma rede sem fio.
\item O sistema deve indicar por meio da vibração a ocorrência de obstáculos a frente do deficiente visual.
\end{itemize}

\subsection{Requisitos de energia}
\begin{itemize}
\item O SmartWay deve ter um sistema de alimentação robusto que tenha autonomia necessária para o uso diário do usuário; 
\item O SmartWay deve um sistema de alimentação com baterias que possam ser recarregada rapidamente;
\item O SmartWay deve ter a entrada de USB adaptado com sinalização em braille, para facilitar a conexão do carregador;
\item O Sistema de alimentação da SmartWay deve ser leve e de dimensão coerente com o design da bengala, para facilitar a adaptação do usuário; 
\item O SmartWay deve ter um aviso sonoro indicando as baterias estão descarregadas;

\end{itemize}

\subsection{Requisitos de software}

A solução de software se concentra de forma geral em dar feedbacks e avisos do ambiente externo ao usuário. Além disso, a fim de melhorar a experiência do usuário que enxerga muito pouco ou nada, será implementado um aplicativo que permite o controle sobre a interação com os componentes de feedback por escuta ou por vibração.

\subsubsection{Reconhecimento de faixa de pedestre}

O Reconhecimento de faixa de pedestre será realizado com o uso de uma câmera acoplada a um óculos. Essa câmera vai capturar os frames de vídeo da faixa de pedestre, que será interpretada por um algoritmo de software.

Além da faixa de pedestre, haverá a instrução de orientação para prosseguir em segurança. Essa instrução consiste em:
\begin{enumerate}
    \item Esquerda: Usuário deverá se deslocar para a esquerda para que possa atravessar em cima da faixa;
    \item Direita: Usuário deverá se deslocar para a direita para que possa atravessar em cima da faixa;
    \item Prosseguimento: Intrução positiva de que pode seguir atravessando a pista.
\end{enumerate}

\subsubsection{Feedback com emissão de mensagem sonora}

O \textit{Smart Way} terá um sistema auditivo, um fone de ouvido ou auto falante, que emitirá mensagens sonoras como feedback do ambiente em que o usuário se encontra. As mensagens sonoras corresponderão à existência de objetos próximos e à frente da bengala e existência de uma faixa de pedestre próxima dele.

O \textit{Smart Way} não emitirá mensagem dizendo qual o tipo do objeto está à frente, se é uma casa, carro, degrau, ou outros. Apenas avisará se existe algum obstáculo. Também não detectará a existência ou não de buracos no chão. O movimento mecânico do usuário com  bengala, de maneira comum, será a maneira de identificar a existência de um buraco ou depressão próximos ao usuário.

O \textit{Smart Way} também irá notificar quando a bateria estiver descarregada.

A API pública para serviços do \textit{Smart Way} deverá ser disponibilizada por apenas para desenvolvedores credenciados que possuem um token no protocolo JWT.

\subsubsection{Aplicativo que controla}

O \textit{Smart Way} será integrado a um aplicativo na plataforma Android. O aplicativo \textit{Smart Way} será um gerenciador de acessibilidade com a bengala e sistema auditivo.

Dessa forma, o usuário será capaz de escolher se deseja ter feedback sonoro ou vibratório (pela pulseira) além de outras funcionalidades.

\section{Requisitos não funcionais}

\begin{itemize}
    \item \textbf{Usabilidade}: O sistema deve ser intuitivo e adaptado para que usuários com deficiência visual possa utilizá-lo da melhor forma possível e a qualidade do acabamento deve ser boa, robusta e ter durabilidade.
    \item \textbf{Manutenibilidade}: O sistema deve seguir as boas práticas utilizadas para desenvolvimento de software.
    \item \textbf{Confiabilidade}: O sistema deve ser testado antes de ir para as mãos do cliente.
    \item \textbf{Desempenho}: O sistema deve ser rápido suficiente para suprir as necessidades do cliente. Não devendo haver tempo de resposta de avisos superior a 1,5 segundos.
    \item \textbf{Portabilidade}: O sistema deve ter um container único de instalação para desenvolvimento e produção.
    \item \textbf{Reusabilidade}: Os recursos do software poderão ser acessado via API pública.
    \item \textbf{Segurança}: O sistema deve ser seguro o suficiente para que o cliente não adquira informações equivocadas;
\end{itemize}

\section{Premissas e restrições}

\begin{itemize}
    \item A faixa de pedestre não deverá ter muitas pessoas transitando sobre ela;
    \item Para a detecção da faixa por imagem, deverá ser em local iluminado;
    \item A qualidade da imagem deverá ser suficiente para que haja uma melhor detecção;
\end{itemize}