\chapter[Introdução]{INTRODUÇÃO}

Deficiência visual é um termo comumente usado para pessoas que são totalmente cegas ou que ainda têm uma visão residual, mas que não podem arcar com ela totalmente. A ocorrência dos problemas no sistema visual pode ser causada por muitas coisas. Alguns deles nascem em estado de cego, acidente, doença, etc. A pessoa com cegueira geralmente usa uma bengala para caminhar ou ir a algum lugar, como um guia para conhecer a direção e conhecer a condição da estrada que está passando. No entanto, as funções da própria bengala convencional ainda são limitadas em informar o obstáculo a um deficiente visual, especialmente quando estão caminhando para o destino remoto \cite{Mutiara2016}. 

As pessoas que têm deficiência visual precisam de uma ferramenta para possibilitar seu caminho seguro para realizar suas atividades. Bastões e bengalas são uma das mais importantes ferramentas de identificação e mobilidade para esse tipo de pessoa. Há muitas bengalas que já estão disponíveis no mercado, mas infelizmente não há muitas variantes, especialmente as bengalas que podem determinar a direção com base na direção do vento e fornecer informações para os cegos sobre a localização da posição deles.

Grande parte deste público deve contar com a assistência de pessoas com visão para encontrar o caminho ou precisar de um acompanhante para seguir; pelo menos durante um período de treinamento. Isso significa que a maioria das pessoas com deficiência visual não consegue encontrar seu caminho de forma autônoma em uma área desconhecida. Nesse caso, costuma-se utilizar uma bengala para auxílio na locomoção. Ela é um dispositivo mecânico puro dedicado a detectar obstáculos estáticos no solo, buracos, superfícies irregulares, degraus e outros perigos através de feedback simples de força tátil \cite{Sakhardande2012}.

Conforme declarado pela OMS em 2014: “estima-se que 285 milhões de pessoas sejam deficientes visuais em todo o mundo: 39 milhões são cegas e 246 milhões têm baixa visão” e “globalmente, os erros de refração não corrigidos são a principal causa de deficiência visual moderada e grave; As cataratas continuam a ser a principal causa de cegueira em países de média renda e de baixa renda ”. A Organização Mundial de Saúde estima que 43\% (123 milhões) deste ônus da doença se deve a URE (sigla em inglês URE - \textit{Uncorrected Refractive Error}) - (miopia, hipermetropia ou astigmatismo) \cite{Contreras2017}.

Sabe-se também que a maioria das pessoas com deficiência visual (90\%) vem de países em desenvolvimento e que 65\% das pessoas com deficiência visual têm 50 anos ou mais \cite{Contreras2017}. 

A necessidade de dispositivos de tecnologia assistiva foi e será constante. Existe uma ampla gama de sistemas de navegação e ferramentas disponíveis para que pessoas com deficiência visual possam ter mais qualidade de vida e autonomia.

A bengala branca e o cão-guia são os exemplos mais populares de auxiliadores de deficientes visuais e cegos. A bengala branca é a mais simples, mais barata, mais confiável e, portanto, a mais popular ferramenta de navegação. No entanto, ele não fornece todas as informações necessárias, como velocidade, volume e distâncias, que normalmente são coletadas pelos olhos e são necessárias para a percepção e o controle da locomoção durante a navegação.

As bengalas brancas são universalmente reconhecidas como símbolos de pessoas cegas, e elas têm sido usadas desde a década de 1920 como auxiliares de mobilidade para guiar os usuários enquanto caminham ou navegam particularmente em lugares desconhecidos \cite{NAP1011}.
Além de sua função básica, que é dar aos usuários informações táteis sobre o ambiente, como obstáculos no chão, buracos e superfícies irregulares, a maioria deles são projetados de tal forma que são leves e retráteis ou dobráveis, o que pode aumentar a conveniência da viagem dos usuários \cite{Ahmad2018}.

No entanto, essas acessibilidades de viagem tradicionais têm apenas faixas curtas de detecção que limitam as detecções de obstáculos abaixo dos níveis do joelho e exigem que as bengalas se coloquem fisicamente nos objetos para alertar os usuários. Devido a essas limitações, as ajudas eletrônicas de viagem (conhecidas em inglês como ETAs - \textit{Electronic Travel Aids}) foram introduzidas nos anos 70, não apenas para estender o alcance do sensoriamento, mas também para promover uma experiência de caminhada independente mais segura e confiante \cite{Dakopoulos2010}.

Segundo \citeonline{Megalingam2014}, a bengala branca é o auxiliar de mobilidade mais comum para os deficientes visuais. No entanto, não fornece informações sobre os obstáculos acima do nível do joelho e aqueles que estão a uma distância maior que 1m. Embora os cães-guia fossem os primeiros acompanhantes dos cegos, mais tarde as tecnologias desempenharam um papel vital. Bengalas com comprimento ajustável, cotovelo, foram desenvolvidas no mercado para orientar os deficientes visuais. No entanto, essas tentativas não foram completamente bem-sucedidas na assistência ao usuário.

A ciência e a tecnologia sempre tentam facilitar a vida humana, havendo uma grande necessidade de meios para ajudar as pessoas com deficiência visual a viver sem problemas relacionados à caminhada entre casas, vias públicas ou em qualquer outro lugar. Mesmo para os não deficientes visuais, o congestionamento de obstáculos às vezes é
problemático, é ainda pior para os deficientes visuais \cite{Therib2017}.

Para auxiliar nestes problemas, o Smart Way é projetado de tal forma que inclui um módulo de Detecção de Obstáculos habilitado para Bluetooth, no qual as informações de distância da placa do microcontrolador são enviadas para o fone de ouvido Bluetooth. Enquanto o usuário recebe feedback de voz sobre os obstáculos estáticos, os motores vibratórios são usados para informar sobre os obstáculos em movimento. A intensidade da vibração depende da velocidade dos obstáculos em movimento. Há também uma integração do módulo com um acessório embarcado, que permite detecção por processamento de imagem de elementos importantes para deslocamento do pedestre, como por exemplo a faixa de pedestre.