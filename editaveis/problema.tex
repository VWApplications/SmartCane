\chapter[Problema]{PROBLEMA}

As bengalas são dispositivos que vem auxiliando os deficientes visuais para que eles possam ter autonomia para desenvolver suas atividades diárias, um dos grande gargalos da bengala é o leque de opções que elas podem transmitir para o deficiente visual, ou seja, uma bengala clássica é um instrumento mecânico que dá uma noção de obstáculos abaixo do nível do joelho. Um deficiente visual não infere se uma pessoa está na sua frente ou a distância, não consegue inferir se chegou em uma faixa de pedestre sem um piso tátil, não consegue saber qual o destino de um ônibus sem ter que perguntar para alguém que está ao seu lado.
	
	Visando essas questões, há a necessidade de que o deficiente visual tenha uma maior autonomia, sabendo que a tecnologia já proporciona diversas melhorias que podem ser acopladas a esse mecanismo.
	
	Hoje em dia, já temos microcontroladores, sensores de presença e sonares, câmeras, processamento de imagem e reconhecimento de locais. Todas essas tecnologias podem ser acopladas a uma bengala branca e podem tornar a vida de um deficiente visual autônoma. 

\section{Justificativa}

	A Universidade de Brasília, junto com alunos da engenharia aeroespacial, engenharia eletrônica, engenharia de software, engenharia de energia e engenharia automotiva, tem o objetivo de desenvolver uma bengala branca inteligente, o desenvolvimento se dará na matéria de Projeto Integrador de Engenharia 2, da Faculdade Gama e será apresentado a comunidade externa na FIT (Feira de Inovação Tecnológica). A bengala (\textbf{\textit{Explicar como será a bengala e os componetes adicionados}}).
	
	No mercado já temos diversas Bengalas inteligentes, a maioria importadas, desenvolvendo uma bengala na Universidade de Brasília, o projeto pode ser aproveitado por diversos Institutos e Associações da região, contemplado o Distrito Federal e entorno.
	

\section{Objetivos}

	O objetivo geral do projeto é desenvolver uma Bengala inteligente capaz de detectar objetos e avisar o deficiente visual através de um fone passar essas informações. 
	
	Os objetivos específicos são detectar objetos, detectar buracos, detectar faixas de pedestres e  informar o destino de ônibus.

\section{StakeHolders}

\begin{table}[h]
	\centering
	\label{tab01}
	\caption{Tabela com informações dos desenvolvedores}
	\begin{tabular}{p{6.5cm}|p{3.2cm} | p{5cm}}
		\toprule
		\textbf{NOME} & \textbf{ENGENHARIA} & \textbf{E-MAIL}\\ \hline
        Guilherme Henrique Rodrigues Vaz & Aeroespacial  & vazghr@gmail.com\\ \hline
          Laís Rocha Carvalho   &  Aeroespacial &  \\ \hline
          Diego Ribeiro Borges Rasmussen & Aeroespacial   &  \\ \hline
         Laís Almeida Nunces & Aeroespacial & laisalmeidanunes@gmail.com \\ \hline
         Lucas Ferreira Rossi & Aeroespacial &  \\ \hline
         Stefânia Bezerra da Silva & Software & \\ \hline
         Wesley Pereira Araujo & Software & \\ \hline
         Victor Hugo Arnaud Deon & Software &  \\ \hline
         Renata Francelino de Souza & Software &  \\ \hline
         Abia Matos de Oliveira & Energia & \\ \hline
         Priscilla Costa de Souza & Eletrônica & priscillacostadesouza@gmail.com\\ \hline
         Victor Barreto Batalha & Eletrônica & victor.batalha@hotmail.com\\ \hline
         Luiza Carneiro Cezário & Eletrônica & \\ \hline
         Luiz Henrique Rocha Marinho & Eletrônica &\\ \hline
		\bottomrule
		
	\end{tabular}
\end{table}


\section{Escopo}

\section{Descrição do projeto}
