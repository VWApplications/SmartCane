\chapter[Termo de abertura]{Termo de abertura}
Este capítulo tem como objetivo dar uma visão geral do projeto.

\section{Descrição do projeto}

Para descrever o projeto, será utilizado a ferramenta 5W2H, um acrônimo em inglês que se refere às principais perguntas que devem ser respondidas no processo de definição.
A seguir podemos melhor relatar o projeto estudado neste trabalho:
\begin{itemize}

    \item \textbf{What?}  Bengala inteligente com acessório com câmera, integrados ao celular do usuário.
    \item \textbf{Why?} Para dar mais acessibilidade a pessoas com deficiência visual na sua locomoção.
    \item \textbf{Where?} Locomoção \textit{outdoor}.
    \item \textbf{When?} Meados de março até início de julho.
    \item \textbf{Who?} Grupo composto por alunos da disciplina Projeto Integrador 2, da Faculdade do Gama - Universidade de Brasília.
    \item \textbf{How?} Utilizando tecnologias das cinco engenharias dispostas no grupo. Uso de sistema embarcado e microcontrolado para ser o cérebro da bengala inteligente,
    controlando o subsistema eletrônico. Uso de sensores e tratamento do sinal de saída para que se torne um sinal mecânico percebido pela pele (sinal tátil) ou sonoro. Uso de câmera e Raspberry Pi, com programação que realizará processamento de imagem por vídeo. Uso de programação para oferecer uma aplicação Android para o celular do usuário. Uso de um sistema portátil, leve (em comparação com as oferecidas pelo mercado), hipermeável e ergonomicamente pensado para proporcionar um conforto para o usuário. 
    \item \textbf{How much?} O custo total estimado ficou em R\$ 830,00, como pode ser melhor visualizado em~\ref{subsec:Registro}
\end{itemize}


\section{Problema}

As bengalas são dispositivos que vêm auxiliando os deficientes visuais para que eles possam ter maior autonomia para desenvolver suas atividades diárias. Uma bengala clássica é um instrumento mecânico que dá uma noção de obstáculos abaixo do nível do joelho. Um deficiente visual não infere se uma pessoa está na sua frente ou a distância, não consegue inferir se chegou em uma faixa de pedestre sem um piso tátil, não consegue saber qual o destino de um ônibus sem ter que perguntar para alguém que está ao seu lado. 

Essas dificuldades podem ser solucionadas de forma popular com um cão-guia, no entanto o adestramento pode custar cerca de trinta e cinco mil reais e a pessoa que faz o pedido pode esperar até três anos para recebê-lo. Essa opção é tão complicada que para o meio milhão de cegos que existem no Brasil, temos apenas cento e cinquenta cães-guia sendo utilizados. Além do custo e da dificuldade de obtenção, há também o processo de adaptação que nem sempre acontece de forma positiva.
	
Diante disso, há a necessidade de que o deficiente visual tenha uma maior acessibilidade ao se locomover. Sabendo que a tecnologia já proporciona diversas melhorias para este público, é do interesse deste projeto ajudar na implementação delas na sociedade brasileira. Já temos microcontroladores, sensores de presença e sonares, câmeras, processamento de imagem e reconhecimento de locais. Tudo isso pode ser acoplado a uma bengala branca tradicional e proporcionar uma vida mais autônoma a um deficiente visual.

\section{Justificativa}

A Universidade de Brasília, junto com alunos da engenharia aeroespacial, engenharia eletrônica, engenharia de software, engenharia de energia e engenharia automotiva, tem o objetivo de desenvolver um projeto de acessibilidade para deficientes visuais. O desenvolvimento se dará na matéria de Projeto Integrador de Engenharia 2, da Faculdade Gama e será apresentado à comunidade externa na FIT (Feira de Inovação Tecnológica). O projeto, nomeado de  \textit{Smart Way}, contemplará uma bengala branca inteligente com acessório de acessibilidade com câmera para processamento de imagens pré-definidas. Para maior integração do usuário com a tecnologia será disposto um aplicativo \textit{mobile}.

No mercado já existem diversas bengalas inteligentes, no entanto a maioria é importada. Ao ser desenvolvido essa tecnologia assistiva na Universidade de Brasília, poderá ser feito um aproveitamento para diversos Institutos e Associações da região, contemplado o Distrito Federal e entorno. 
	

\section{Objetivo}

O objetivo geral do projeto é desenvolver uma tecnologia assistiva - que inclui uma bengala inteligente e um acessório com câmera - capaz de detectar objetos e avisar o usuário através de sinais táteis e sonoros.

Os objetivos específicos são com este produto detectar objetos, detectar buracos, detectar faixas de pedestres e através de uma aplicação \textit{mobile}, o usuário poderá configurar o sistema para receber avisos sonoros, vibratórios entre outras funcionalidades.
